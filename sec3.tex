%\documentclass[biologie_praktikum_2+3.tex]{subfiles}

%\begin{document}

\section[Morphologische Eigenschaften \& GRAM-Färbung] {Herstellung eines Ausstrichpräparates einer ausgewählten Bakterienkolonie der Fangplatte und GRAM-Färbung zur Mikroskopie}
\label{sec:Gram}

\subsection{Versuchsdurchführung}
\label{sec:Gram-V}
Es wurde sich an die Arbeits- und Sicherheitsvorschriften gehalten.

\subsection{Ergebnisse}
\label{sec:Gram-E}

\begin{table}[h]
\centering
\caption{Ergebnisse \ref{sec:Gram-E} \\ 
Morphologische Eigenschaften}
\label{Tab:Gram-E}
\begin{tabular}{ll}
\hline
Eigenschaften & Wert    \\ \hline
Farbe         & Gelb    \\
Form          & Rund    \\
Profil        & Erhaben \\
Konsistenz    & Buttrig \\
Geruch        & /       \\
Kolonienrand  & Glatt   \\ \hline
\end{tabular}
\end{table}

Die Form ist Diplokokkoid.
Die GRAM-Färbung ergab eine Gram-Positive art.

Die Skizze ist Anhang A.

\subsection{Auswertung}
\label{sec:Gram-A}

Auf grund der stark Dunkel-blauen Färbung ist von einer Gram-Positiven Art auszugehen.

%\end{document}