%\documentclass[biologie_praktikum_2+3.tex]{subfiles}

%\begin{document}
%-----------------------------------------------------------------------------------
%Fangplatten
%----------------------------------------------------------------------------
\section{Exposition und Beimpfen der Nähragarplatten}
\label{sec:Exp}

\subsection{Luftfangplatte}
\label{sec:Exp-Luft}

\subsubsection{Versuchsdurchführung}
\label{sec:Exp-Luft-V}
Die Agarplatte wurde nach Vorschrift, am mit dem Betreuer ausgemachten Ort
(Fensterbank außen) 30 Minuten exponiert.
Die Nährmedien wurden dann bis zum nächsten Praktikumstermin aufbewahrt, in dem die Auswertung erfolgte.    
\subsubsection{Ergebnisse}
\label{sec:Exp-Luft-E}

%Tabelle 
\begin{table}[b]
\centering
\caption{Ergebnisse \ref{sec:Exp-Luft-E}}
\label{Tab:2.1.2}
\begin{tabular}{c|c}
Expositionsort & \begin{tabular}[c]{@{}c@{}}Kolonie Bildende\\ Einheiten\end{tabular} \\ \hline
Sterilwerkbank & 0 \\
\begin{tabular}[c]{@{}c@{}}Labor\\ Innen\end{tabular} & 13 \\
\begin{tabular}[c]{@{}c@{}}Fensterbank\\ aussen\end{tabular} & 18 \\
Kühlschrank & 11 \\
Kühlschrank & 10 \\
Toilette & 2 \\ \hline
\end{tabular}
\end{table}
%

In unterschiedlichen Expositionsorten sind verschieden viele Mikroorganismen gewachsen.
So sind bei den Nährmedien vom Labor und der äußeren Fensterbank die meisten Mikroorganismen (13 \& 18) KBE vorhanden.
Der Kühlschrank wurde einer Doppelbestimmung unterzogen, welche ähnliche werte (10 \& 11 KBE) zueinander aufweisen.
Das Nährmedium was in der Sterilwerkbank exponiert wurde hatte keine Koloniebildende Einheit.
Beim Expositionsort Toilette haben sich 2 KBE gebildet.

Nach Beimpfen und wachsen der Mikroorganismen wurden diese ausgewertet, siehe 	\ref{sec:Exp-Luft-A}~\nameref{sec:Exp-Luft-A}.

\subsubsection{Auswertung}
\label{sec:Exp-Luft-A}

Da sich an einem Ort draußen sich durch die luftbewegung sich die meisten Mikroorganismen aufhalten, ist zu erwarten das sich dort die meisten KBE bilden.
Diese erwartung wurde bestätigt.

Insbesondere dadurch, dass sich im labor innen 13 KBE gebildet haben, welche die zweithöchste anzahl ist.
Die liegt daran, das am Tag der Versuchsdurchführung, das Fenster längere zeit geöffnet war, wodurch sich eine Luftzirkulation eingestellt hat.

Die ergebnisse der doppelbestimmung des Kühlschranks zeigen keine große abweichung voneinander.

Man könnte annehmen, das der Expositionsort Toilette, viele KBE ausbildet. Jedoch sind nur 2 KBE nachzuweisen.
Dies könnte an der wenigen Zirkulation, und dadurch verringerten bewegung der Mikroorganismen in der Luft, führen.
Zudem, wird die Toilette regelmäßig gereinigt.

Die Funktion der Sterilwerkbank war gewährleistet.
Durch den stetigen Lufstrom gelang es keinen Mikroorganismus, eine Kolonie auf dem Nährmedium zu bilden.

Die wirkungen von Reinigungsmittel werden in \ref{sec:Exp-Des-A} besprochen.
%-----------------------------------------------------------------------------
%Reinigungsmittel
%------------------------------------------------------------------------------
\subsection{Untersuchung der Widerstandsfähigkeit von Mikroorganismen in Bezug auf Desinfektions- und Reinigungsmitteln}
\label{sec:Exp-Des}

\subsubsection{Versuchsdurchführung}
\label{sec:Exp-Des-V}

Es wurde sich an die Versuchs- und Arbeits-vorschriften gehalten.

\subsubsection{Ergebnisse}
\label{sec:Exp-Des-E}
Die zwei viertel in welche die unbehandelten Finger gegeben wurde, hatten in der abdrucksfläche ein Wachstum zu verzeichnen.
Beide hatten ein ähnliches Wachstum an Mikroorganismen.

Beim Viertel der mit seife behandelten Finger waren weniger Mikroorganismen zu erkennen.
Gegenüber den unbehandelten Flächen war die Hälfte an Organismen zu 
beobachten.

Das Viertel, wo die Hände mit sterillium behandelt wurde, hatte keine gewachsenen Mikroorganismen zu verzeichnen.


\subsubsection{Auswertung}
\label{sec:Exp-Des-A}

Unsere erwartung spiegeln sich in dem Ergebnissen wieder.
Die seife wirkt, in dem der Staub, an dem die Mikroorganismen haften, von der seife umschlossen und fortgespült wird.
Das Ergebnis ist zufriedenstellend.
Fast die Hälfte der Mikroorganismen wurde entfernt gegenüber dem wert der unbehandelten Flächen.

Desinfektion beseitigt nach Definition 10$^5$ Mikroorganismen.
Unser Ergebnis von der mit Sterilium behandelten Fläche hat dies bestätigt.

%--------------------------------------------------------------------------------
% UV-Strahlung
%-------------------------------------------------------------------------------
\subsection{Widerstandsfähigkeit von Micrococcus sp. und Bacillus subtillis bei Exposition unterschiedlicher Dauer von UV-Strahlung}
\label{sec:Exp-UV}

\subsubsection{Versuchsdurchführung}
\label{sec:Exp-UV-V}

Es wurde sich an die Versuchs- und die Arbeitsvorschriften gehalten.

\subsubsection{Ergebnisse}
\label{sec:Exp-UV-E}

\paragraph*{Bacillus Subtillis:}
Alle Kontrollkulturen wiesen einen geschlossenen Bakterienrasen auf.
Bei einer Zeit von 60 Sekunden waren viele Einzelkolonien zu verzeichnen, 90 Sekunden eine verminderte Anzahl an Kolonien, und bei 150 Sekunden wenige Einzelkolonien mit einer mäßigen Ausbreitung.

\paragraph*{Micrococcus spp:}
Alle Kontrolkulturen wiesen einen geschlossenen Bakterienrasen auf.
Die 60 sekunden mit UV-Strahlung bestrahlte kultur zeigte Viele Einzelkolonien.
Bei einer 90 sekunden langen bestrahlung wuchsen mäßig Viele Einzelkolonien.
150 sekunden lange bestrahlung hatte den effekt, das nur wenige einzelkolonien wuchsen.
%Tabelle 2
\begin{table}[bp]
\centering
\caption{Ergebnisse \ref{sec:Exp-UV-E}}
\label{tab:Exp-UV-E}
\begin{tabular}{lll}
 & UV-bestrahlt & Kontrolle \\ \hline
Expositionszeit & \multicolumn{2}{l}{Bacillus Subtillis} \\ \cline{2-3} 
60 s & Viele Einzelkolonien & geschlossener Bakterienrasen \\
90 s & mäßig Viele Einzelkolonien & geschlossener Bakterienrasen \\
150 s & wenige Einzelkolonien & geschlossener Bakterienrasen \\ \hline
 & \multicolumn{2}{l}{Micrococcus sp.} \\ \cline{2-3} 
60 s & Viele  Einzelkolonien & kein geschlossener Bakterienrasen \\
90 s & mäßg Viele Einzelkolonien & geschlossener Bakterienrasen \\
150 s & wenige Einzelkolonien & geschlossener Bakterienrasen \\ \hline
\end{tabular}
\end{table}
%

\subsubsection{Auswertung}
\label{sec:Exp-UV-A}


Die UV-strahlung hatte einen antiproportionalen effekt auf beide Mikroorganismen.
Mit verlängerter dauert wuchsen weniger bakterienkolonien.
Die Einzelkolonien erklären sich , weil ein großteil der Bakterien während der Bestrahlung abgestorben sind.\par

Wir sind davon ausgegangen, das Bacillus Subtillis, durch seine Sporenbildende Eigenschaft,
eine höhere resistenz ausweist als Micrococcus spp.
Micrococcus hat uns Erwartungen bestätigt, aufgrund seiner fehlenden Sporenbildung.
\par
Unsere Ergebnisse weichen von unseren Erwartung bei Bacillus Subtillis ab.
Mögliche erklärungen sind, das Lebewesen der selben Art immer geringfügige Unterschiede besitzen.
Dies könnte auch erklären, warum sich bei der Kontrollagarplatte, bei Micrococcus 60 s, kein geschlossener Bakterienrasen gebildet hat.
Theoretisch könnte die Bacillus-art auch eine schwächere restistenz gegenüber UV-Strahlung besitzen.
%\end{document}