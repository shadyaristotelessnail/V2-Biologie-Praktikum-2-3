%\documentclass[biologie_praktikum_2+3.tex]{subfiles}
%\begin{document}
\section{Färbung und Mikroskopie der Blutausstriche von Versuch 1}
\label{sec:Blut}


\subsection{Versuchsdurchführung}
\label{sec:Blut-V}
Es wurde sich an die Arbeits- und Sicherheitsvorschriften gehalten.
Die Praktikumsbetreuerin hat uns angewiesen, die Leukozyten aus dem Vollblutausstrich zu skizzieren.
\subsection{Ergebnisse}
\label{sec:Blut-E}

\paragraph*{Vollblutausstrich}
In unseren Ausstrich waren folgende Zelltypen zu sehen:

\begin{table}[h]
\centering
\caption{Ergebnisse 4.2}
\label{Tab:Blut-E}
\begin{tabular}{ll}
Zelltyp      & geschätzer Anteil  \\ \hline
Erythrozyten & \textgreater{}95\% \\
Leukozyten     & \textless{}1\%     \\
\end{tabular}
\end{table}

Die Erythorozyten haben den größten anteil mit über 95\%.
Leukozyten und nehmenbedeutend weniger als 1\% ein.

\paragraph*{Plättchenreiches Plasma}
Im Plättchenreichem Plasma haben wir nur Thrombozyten mikroskopieren können.
Dementsprechend ist der  geschätze Anteil an Thrombozyten im PRP \textgreater{99}\%

\paragraph*{Weißes Sediment}
Im weißen Sediment konnten keine Zelltypen gefunden werden.
Deswegen sind die geschätzen Anteile gleich dem des Vollblutausstrichs.


\subsection{Auswertung}
\label{sec:Blut-A}

In dem Vollblutausstrich waren zu erwartenden Anteile vorhanden.
Dies bestätigen die Literaturwerte .
\cite{Antwerpes2018}

Da wir zum Skizzieren der Leukozyten den Vollblutausstrich verwenden sollten, kann keine Angabe zu den geschätzen Anteilen der Einzelnen Zelltypen im Weißen Sediment gemacht werden.
Laut Literaturwerten sind Hauptsächlich Leukozyten im Weißen Sediment("Buffy Coat").


Wie zu erwarten waren im PRP viele Thrombozyten zu beobachten.



%\end{document}