\section{Chromosomenpräparat}
\label{sec:Chromos}

\subsection{Versuchsdurchführung}
\label{sec:Chromos-V}
Ein Praktikumsteilnehmer hat für die gesamte Gruppe ein Chromosomen-präparat hergestellt.
Die Praktikumsbetreuerin hat das Präparat Mikroskopiert und auf einem Monitor sichtbar gemacht.
\subsection{Ergebnisse}
\label{sec:Chromos-E}

Es waren folgende Formen zu verzeichnen:
\begin{itemize}
\item Chromosomen
\item Chromatin
\item Chromatin in der Kernmembran
\item Geteilte Zelle
\end{itemize}

Die Skizze ist Anhang E.

\subsection{Auswertung}
\label{sec:Chromos-A}
Erst bei 1000x~-facher Vergrößerung konnten die Stadien gut erkannt werden.
Gewisse ausschnitte des Präparates lassen sich nicht zweifelsfrei zu ihren Stadien zuordnen.

